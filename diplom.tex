\documentclass[a4paper, 12pt, onepage, pdftex, headsepline, footsepline]{scrreprt}

\usepackage[utf8]{inputenc}
\usepackage[T1]{fontenc}
\usepackage{setspace}
\onehalfspacing

\usepackage[ngerman]{babel}
\usepackage{natbib}
\usepackage{url} %correct url display in cites

\pagestyle{headings}

\typearea[current]{calc}

%add hyphenation later

\begin{document}
\title{Non-Standard-Datenvisualisierer mit openFrameworks}
\author{Benjamin Knofe}
\subject{Diplomarbeit}
\publishers{Hochschule für Technik, Wirtschaft und Kultur Leipzig}
\dedication{viele nette Leute}
\maketitle
\tableofcontents
\chapter{Einleitung}
%\section{Ein Abschnitt}
\label{sec:testi}

digital gespeicherte daten nehmen zu.
ob moores law, 20 monate verdoppelung der daten oder selbst werbung großer unternehmen (IBM im Spiegel) beweisen das.

1. Motivation / Relevanz des Themas
immer mehr daten, nur custom tools von profis nutzbar, weniger zeit für die einarbeitung
2. problemstellung / ziel der arbeit
einen software prototypen zu entwickeln, der auf basis einer der grundprinzipien der visualisierung beruht: variablen von daten auf variablen der grafik anzuwenden
3. methodisches vorgehen und aufbau
klärung der begriffe/probleme/einordnung in den workflow, festlegen der besten tools und patterns die software zu entwickeln, software beispielhaft implementieren Text.

bild von mccandless mit den datenraten von auge zu hirn

das problem sollte hier geklärt werden:
immer komplexere datenstrukturen können nicht mehr mit klassischen mitteln wie excel einfach dargstellt werden. man braucht einen neuen ansatz (und die dazugehörige software)

um dieses problem zu lösen wird auf einen grundlegenderen und abstrakteren ansatz eingegangen und darauf ein tool entwickelt

bestandteil von UI (siehe PREIM/DACHSELT 506)

auch wenn jede viz konkrete ziele hat, mit denen sehr spezielle probleme gelöst werden, addressieren diese nicht mehr nur Experten und Wissenschaftler. Auch alltägliche Anwendungen wie Viz all meiner Freunde in einem sozialen Netzwerk, beim Einkaufen und suchen von Informationen in der Bib können mit Viz unterstüzt werden und müssen somit auch für ungeübte Anwender verständlich und benutzbar sein. Dabei sollte die Viuz ansprechend sein damit lust auf benutzung besteht aber gleichzeitig auch intuitiv um ohne vorkenntnisse nutzen aus dieser ziehen zu können.
Auch kann die tägliche Arbeit unterstützt werden, damit es Gelegenheitsanwendern leicht gemacht wird, Informationen und Wissen noch besser mit anderen teilen zu können ( VIZ in der Präsi, ...)

Häufig sind die Grenzen der einzelnen Arbeitsbereiche nicht klar voneinander getrennt. Verwandte Arbeitsfelder sind (Visual) Data Mining, ....
Weiterhin ist die Visualisierung im allgemeinen ein Teil anderer Arbeitsbereiche. Zum Beispiel spielt die Vizualisierung im Interface Design eine ebenso wichtige Rolle wie im Data Mining. 

Wichtig: Viz findet im Kopf des Betrachters statt, nicht in der Maschine (grün weißes buch aus campus bib)


\section{Umfeld der Arbeit}
etwas zu mir?
\section{Problemstellung}
hier muss genau das problem geklärt werden, was zur zeit besteht.
es gibt kein tool, dass spezialisiert auf das grundprinzip der viz ist.
es werden viele programme verwendet die aus teilbereichen stammen (oder customlösungen), dadurch erhört sich arbeitsaufwand, schlechtere qualität des ergebnisses, zu viel vorkenntnisse notwendig.

unterhaltungswert des graphen und ästhetik spielen eine rolle, denn man muss was gerne anschauen wollen damit man sich überhuapt damit auseinandersetzt


\section{Aufbau der Arbeit}
Im Kapitel Daten und Informationen soll versucht werden eine Klassifikation dieser beiden Begriffe in Hinblick auf die folgenden Vizualisierungsmethoden genommen werden.
\chapter{Grundlagen}
\section{Daten und Informationen}
was für daten können verwendet werden? wetterdaten? pachube, sämtliche (Web)-Apis
Unterschied zwischen DAten und Information in Bezug auf VIZ!

definition und klassifizierung von daten

Beobachtungsraum, -punkt, -fall, Merkmale, Ausprägung (vielleicht erst später? ), Metadaten

möchte sich diese arbeit mit der spezialform von daten, den abstrakten Daten beschäftigen, die keinen räumlichen und oder zeitlichen bezug haben.
abstrakte daten haben alle anforderungen die auch normale daten haben und deswegen kann datasynth beides

infoviz als arbetisfeld mit noch spezifischeren anforderungen

da informationen eine spezielle form von daten sind, schließen diese nicht nur die Anforderungen an datenviz ein sondern verlangen noch zusätzliche Gedanken

bertin benutzt ausschließlich den Begriff der Information

\subsection{Informationsbegriff nach Bertin???}

klassifizierung von informationen in bezug auf viz

\subsection{multivariate Daten}
\subsection{mehrdimensionale Daten}
\subsection{raumbezogene Daten}
\subsection{zeitbezogene Daten}
statisch, quasistatisch, dynamisch
\subsection{strukturelle Beziehungen zwischen Datenobjekten}

\section{graphische Grundlagen}

klärung und definition aller graphischen grundlagen
graphische semiologie?
was macht eine grafik eigentlich aus.....?

Diagramm?

\section{Arbeitsfeld Visualisierung}
visualisierung im allgemeinen, warum wieso und für was
Abgrenzung zu anderen Bereichen (vielleicht eigener gliederungspunkt) -> data mining.....
Abgrenzung nach innen (ben fry)
\section{verwandte Arbeitsfelder}
\section{(Daten-)Visualisierung als Problemlösungsansatz}
Welches Problem gibt es?
Warum?
Andere Lösungsansätze?
begrifflichkeit klären und sagen man verwendet ab jetzt nur noch viz stellvertretend für info/data-/sci-viz

sehr spezifische und stark anwendungs und erkenntnis gewinn abhängige anforderungen an visualisierung. 
es ist schwer allgemeingültige aussagen zu treffen wie eine Information am besten visualisiert werden kann und somit muss eine Anwendung einen möglichst generischen ansatz bieten damit der benutzer ohne einschränkung seine spezielle aufgabe die er mit der viz lösen will lösen kann.

erkenntnisgewinn und aufgabe die ich lösen möchte. also eine viz hat ja immer ein konkretes problem was der uafgabe zugrunde liegt und mit viz gelöst werden soll. (das nochmal genau klassifizieren)

welche viz eignet sich am besten für meine konkrete aufgabe?
gibt es mischformen von bis jetzt gefundenen viz mit denen ich meine aufgabe noch besser lösen kann?
gibt es neue ansätze für viz die bis jetzt noch nicht beschrieben sind. (unterstützung der forschung nach neuen formen der viz)

Was ist das Interaktionsziel? (Begriff klären)

diese arbeit bezieht sich auf informationsvisualisierung als sonderfall der datenvisualisietrung. damit werden alle dinge der datavz beachtet und zusätzlich alle sonderfälle der infoviz beachtet

\section{Was macht eine gute Viz aus?}
\section{(Daten-)Visualisierung als Arbeitsprozess}
warum braucht man das? welche probleme werden damit gelöst?
welche anforderungen? (in welchen bereichen des täglichen lebens, der wirtschaft)
\subsection{Abgrenzung (nach Ben Fry)}
einordnung an der einteilung von ben fry
\subsection{Anwendungsgebiete von Viz}
wissenschaftlich, technisch, ....
\section{Grundprinzipien der Datenvisualisierung}
entwicklung einer kleinen "theorie"
wie können die grundprinzipien der viz auf eine software überrtragen werden?
welche vorteile/nachteile hat das?
welche probleme werden damit gelöst? (keine festlegung mehr auf balken.....)
welche neuen erkenntnisse können damit entstehen? (neue sichtweisen auf daten werden entdeckt, was wiederrum neue sichtweisen des eigentlichen problems aufzeigt)
zugriff und einfluß und übersicht auf alle parameter der geometrie und deren attribute....
dadurch: direkter einfluß aber auch direktes verständis der einzelnen elemente

grafische Transkription (BERTIN)

Auflsitung und kurze Beschreibungen aller möglichen Abbildungen (siehe Schumann S. 126)

als universellen in jeder viz theorie vorkommenden schritt
\subsection{graphische Semiologie}
BERTIN spricht von grapischer Semiologie, also von einem genau festgelegten und endlichen Zeichensystem für die graphische Repräsentation von Information.
Information bedeutet dabei siehe BERtin 13 Somit fasst Bertin den Informationsbegriff etwas allgemeiner und fasst damit Daten ein. (stimmt das?)
\subsection{Mapping}

Verbindung von Variablen der Information mit Variablen der Grafik.
dabei müssen die besonderheiten und merkmale der Komponenten der INFO (siehe bertin) klar sein und müssen auf adäquate visuelle Variablen angewendet werden.
trotz einer nahezu unendlichen auswahl an möglichkeiten zur konstruktion einer grafik sollten ein paar regeln beachtet werden (qualitative klassen nicht auf quantitative variablen usw.)


\subsection{verschiedene Darstellungsformen}
sind alle stark anwendungsabhängig und es muss für jede anforderung festgelegt werden (siehe das kapitel wo steht welche probleme man lösen kann (siehe schumann mit den 3 dingen einer viz, das letzte war kommunikation))
Preim/Dachselt ergänzen zu den von schumann genannten noch die viz von relationen als eigenständige klassifizierung von daten

hier die "datentypen" aufzählen

\section{Konzept einer geeigneten Softwarelösung}
beispielhafte implementierung “datasynth”
anforderung an so eine software (beispiel1: journalist, beispiel2: privat/künstler)
-> datasynth als Erkundungstool für daten, als datenleser (ist das nicht auch eine kompetenz die man entwickeln sollte: daten lesen und verstehen zu können, also heisst es nicht auch heutzutage und in zukunft immer mehr: daten verstehen heißt die welt verstehen)

Warum visuelle Programmierung? DEFINITION! und nochmal WARUM?
weil es dem zugrundeliegenden prinzip des MAPPING der viz entspricht

statt custom tools etc und immer wieder neu programmieren.....jetzt ein tool

komplexität hinter einem einfachen interface verstecken (klavier aus memo’s talk in london)

einordnung der software mit vergleichen zu anderen apps (grundlage sind funktionalität, also was kommt dabei raus) ist so ein ding zwischen statistiktool spss, 

iterativer prozess der exploration wird unterstützt (vgl. mantra visueller informationssuche) (vgl. preim/dachselt, s 443 ganz unten)

um den vielseitigen anforderungen an eine visualisierung gerecht zu werden, soll daher eine softwar entwickelt werden, die auf einem Level ansetzt den jeder Visualisierungsprozess durchlaufen muss: das Mapping bzw graphische Semiologie. Somit kann der gesamte Bereich der Visualisierungsmöglichkeiten theoretisch abgedeckt werden.

man kann jede visualisierung in ihre grundlegenden graphischen elemente zerlegen.
was sind die grundlegenden graphischen elemente? (herausfinden! vielleicht in bertin!)

\section{visuelle programmierung als möglichkeit des UI}
rapid prototyping / keine programmierkenntnisse erforderlich, ein durchschnittliches mathematisches verständnis reicht
theoretische konzeption
erklärung der einzelnen teile wie nodes etc doch erst bei implementierung?
\subection{Spread}
\subsection{Node}
\subsection{Pin}
\subsection{Connection}

\chapter{Grenzen der Software}
\section{technische Grenzen}
pixel (bildschirm)
farbe (rgb)
\section{theoretische Grenzen}
viz von mehr als 3 dimensionen -> projektion
\scetion{biologische Grenzen}
wahrnehmungs- und auflösungsfähigkeit des menschen
verständnis (wieviel kann ein mensch mit einmal überblicken)
\section{Grenzen durch Implementation}
künstliche Grenzen weil ja nur Prototyp entwickelt wird.
statisch, 2D vollständigkeit wird durch dateninput festgelegt (vgl. schumann 6.2.1, s. 175)


\chapter{Methodik}
\section{Anforderungen}
möglichkeit alle informationstypen zu viz0

1. konkretisierung der eigenen problemstellung
software erstellen, was muss sie können, wofür soll sie da sein
also warum z.b. visuelles programmieren usw.
2. entwicklung von hypothesen (oder definition von anforderungen)
was muss die software leisten können?
3. ausführliche beschreibung Untersuchungsmethodik und Vorgehensweise
????

weil mit dem Grundprinzip der Abbildung gearbeitet wird, müssen alle Möglichkeiten dieser in der Software zur Verfügung stehen

keine vorgabe eines bestimmten system sondern die möglichkeit direkt mit grapischen primitiven und ihren attributen wie farb, etc. arbeiten zu können

möglichst viel Parametrisiert um Wertebereiche direkt anpassen zu können

\subsection{cross-platformness}
\subsection{freie Lizenz}
rails als beispiel für MIT Lizenz

\section{Analyse bestehender Lösungsansätze}
diskussion bisheriger lösungsansätze
vorstellen aller software die es bis jetzt gibt, siehe oben
evtl. (SWOT-)Analyse und damit aufzeigen der schwächen der anderen software

es gibt eine vielzahl unterschiedlichster programme und programmbibliotheken die genutzt werden können um daten zu visualisieren.
aus den verschiedenen bereichen: statistik, grafik, ...
es wird hier nur auf eine auswahl eingegangen, die elemente enthalten, die in datasynth sein sollten
\subsection{vvvv}
\subsection{pd - Pure Data}
\subsection{prefuse}
\subsection{graphviz}
\subsection{...}
\section{Definition eines Anforderungskatalogs}
möglichkeit alle datentypen abbilden bzw verarbeiten zu können
bearbeitungsziele sollen schnell und unkompliziert erfüllt werden können
\chapter{Durchführung}
1. dokumentation der untersuchungsdurchführung (des Entwicklungsvorganges)
welche tools wurden wie angewendet, c++, of, git, boost, etc etc
2. darstellung der ergebnisse (verdeutlichung durch beispiele, erläuterung des erkenntnisfortschritts)
präsentation von viz mit der software?!
\section{Technologien}
\subsection{C++}
\subsection{openFrameworks}
\subsection{boost}
\section{wichtige Kernfunktionen}
auszüge aus quelltext mit erklärung
\section{Probleme bei der Entwicklung}
reflection/introspection von c++
universelle datentypen (schwierig in c++)
geeignete daten finden
mulit-window ausgabe (gelöst mit ofxFenster, da glut das nicht hergibt)
\chapter{Auswertung}
bewertung der ergebnisse vor dem hintergrund der hypothesen/problemstellung (Test des konstrukts und vergleich mit den Anforderungen
was kann die software liefern? wofür bildet sie nur die grundlage? was kann sie nicht?

2. ableiten von schlußfolgerungen
????
so eine software macht sinn, macht keinen sinn?
\section{Vorteile}
durhc parameter kann der benutzer die Wertebereiche der Abbildungen direkt verändern und kann somit auch ohne komplette kenntnis der daten live die beste "einstellung der parameter" herauszufinden, anstatt vorher Metadaten kennen bzw studieren bzw anlegen zu müssen (min,max, ...)

mehrere viz können statisch in einer präsentation kombiniert werden. so kann ohne benutzerinteraktion mehrere blickwinkel auf den datensatz ermöglicht werden

\section{Nachteile}
benutzer hat alle freiheiten aber muss dadurch auch die grundregeln der viz beherrschen und verstehen ( z.b. kreis radius/fläche, was eignet sich für was am besten)
gestaltungsregeln müssen vom nutzer angwendet werden um viz zu verstärken
Klärung der möglichen Fragen, use nach BERTIN (Theorie des graphischen Bildes)

erweiterbarkeit:
wenn eine gewisse funktionalität noch nicht als node vorliegt, kann diese hinzuprogrammiert werden. verweis dabei auf die angedachten features zur erleichterung von nutzer entwicklungen mit beispielsweise ruby

das statische:
fehlende nutzerinteraktion und dadurch fehlende interaktive kombination von viz techniken (klick auf glyphe erzeugt eine genaue viz der werte dieses datenobjektes)

low-level-ansatz:
führt dazu, dass man immer vom urschleim anfangen muss um komplexe objekte zu erzeugen. deshalb wichtiges feature: kapselbarkeit von mehreren nodes zu einer komplexen high-level-node
so könnte man z.b. eine Sunburst node kreiiren die dann zwar 10 anstöpselmöglichkeiten hat aber sonst nix weiter
man benutzt grafische primitive und manche komplexe viz hat sehr viele primitive in komplexen abhängigkeiten, also kapseln

\chapter{Zusammenfassung}
1. zusammenfassung der ergebnisse, lesson learned
???
2. optional: persönliche bemerkungen, hinweis auf weiteren forschungsbedarf, ausblick in dei zukunft
was kannnoch alles an der Software getan werden?
\section{Ausblick}
\subsection{weitere Features}
ruby
erweiterung auf andere bereiche des visualisierungsprozesses
kapselung von patches
  vielfältige möglichkeiten wie
    erweiterung und kapselung von verschiedenen viz templates als nodes (Isofläche Node oder Flow Ribbons)
    eigene erstellung von ikonen /vgl. schumann s. 197)
    (dazu erfüllt datasynth alle genannten anforderungen von sich aus (editor, binder, viewer)

v4p laden und speichern möglich! crazy!

erweiterung der ausgabemöglichkeiten (statt ausgabe auf bildschirm, ausgabe als postscript, ausgabe als interaktive 3D HTML5 Anwendung (dazu wären navigations und interaktionsmöglichkeiten notwendig (vgl. schumann s. 175), ...)

eng verbunden damit sind die möglichkeit allgemeine interaktionsmöglichkeiten einzubauen um parameter für beispielsweise selektionsbedingungen auch von end-anwendern der viz benutzbar zu machen

evaluationsmöglichkeit (wie kann ich schon in datasynth testen ob die ausgabe geeignet ist) (muss erforscht werden)

automatische generierung von legenden/skalen(+einteilungen), muss erforscht werden inwieweit das überhaupt machbar ist bei einem so universellen ansatz oder ob es nicht doch besser ist, diese dinge der nachbearbeitung zu überlassen (weil datasynth ja nur die massen-arbeit machen soll und nicht einzelne details am ende)

eventbasierte steuerung

threaded nodes

reiner export als abstrakte werte (koordinaten, höhe, breite, typ) in gängige 3D-Austausch-Formate

\section*{Viz und so}

Text über viz mit einer Fußnote\footnote{Streng genommen eine ziemlich sinnfreie.}

\section*{Querverweise}

Aber er hat einen coolen Verweis, siehe \cite[S.\,35--38]{Visualisierung}.
Mal sehen wie das in einer Fußnote\footnote{\cite{Visualisierung}} aussieht.

Um Querverweise nutzen zu können muss man z.B. auf Abschnitt~\ref{sec:testi} verweisen.

\bibliographystyle{dinat}
\bibliography{diplom}
\listoftables
\listoffigures
%\printnomenclature
\chapter{Anhang}
\section{Beispiel: Bevölkerungsdaten Stadt Leipzig}
viz mit streckenzügen (sternförmig, parallel), kreisen (matrix),
\end{document}

