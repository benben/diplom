\documentclass[a4paper, 12pt, onepage, pdftex, headsepline, footsepline]{scrreprt}

\usepackage[utf8]{inputenc}
\usepackage[T1]{fontenc}
\usepackage{setspace}
\onehalfspacing

\usepackage[ngerman]{babel}
\usepackage{natbib}
\usepackage{url} %correct url display in cites

\pagestyle{headings}

\typearea[current]{calc}

%add hyphenation later

\begin{document}
\title{Non-Standard-Datenvisualisierer mit openFrameworks}
\author{Benjamin Knofe}
\subject{Diplomarbeit}
\publishers{Hochschule für Technik, Wirtschaft und Kultur Leipzig}
\dedication{viele nette Leute}
\maketitle
\tableofcontents
\chapter{Einleitung}
%\section{Ein Abschnitt}
\label{sec:testi}

1. Motivation / Relevanz des Themas
immer mehr daten, nur custom tools von profis nutzbar, weniger zeit für die einarbeitung
2. problemstellung / ziel der arbeit
einen software prototypen zu entwickeln, der auf basis einer der grundprinzipien der visualisierung beruht: variablen von daten auf variablen der grafik anzuwenden
3. methodisches vorgehen und aufbau
klärung der begriffe/probleme/einordnung in den workflow, festlegen der besten tools und patterns die software zu entwickeln, software beispielhaft implementieren Text.

bild von mccandless mit den datenraten von auge zu hirn

das problem sollte hier geklärt werden:
immer komplexere datenstrukturen können nicht mehr mit klassischen mitteln wie excel einfach dargstellt werden. man braucht einen neuen ansatz (und die dazugehörige software)

um dieses problem zu lösen wird auf einen grundlegenderen und abstrakteren ansatz eingegangen und darauf ein tool entwickelt
\section{Umfeld der Arbeit}
etwas zu mir?
\section{Problemstellung}
\section{Aufbau der Arbeit}

\chapter{Grundlagen}
\section{Daten und Informationen}
was für daten können verwendet werden? wetterdaten? pachube, sämtliche (Web)-Apis
\section{Arbeitsfeld Visualisierung}
visualisierung im allgemeinen, warum wieso und für was
Abgrenzung zu anderen Bereichen (vielleicht eigener gliederungspunkt) -> data mining.....
Abgrenzung nach innen (ben fry)
\section{verwandte Arbeitsfelder}
\section{(Daten-)Visualisierung als Problemlösungsansatz}
Welches Problem gibt es?
Warum?
Andere Lösungsansätze?
begrifflichkeit klären und sagen man verwendet ab jetzt nur noch viz stellvertretend für info/data-/sci-viz
\section{(Daten-)Visualisierung als Arbeitsprozess}
warum braucht man das? welche probleme werden damit gelöst?
welche anforderungen? (in welchen bereichen des täglichen lebens, der wirtschaft)
\subsection{Abgrenzung (nach Ben Fry)}
einordnung an der einteilung von ben fry
\section{Grundprinzipien der Datenvisualisierung}
entwicklung einer kleinen "theorie"
wie können die grundprinzipien der viz auf eine software überrtragen werden?
welche vorteile/nachteile hat das?
welche probleme werden damit gelöst? (keine festlegung mehr auf balken.....)
welche neuen erkenntnisse können damit entstehen? (neue sichtweisen auf daten werden entdeckt, was wiederrum neue sichtweisen des eigentlichen problems aufzeigt)
zugriff und einfluß und übersicht auf alle parameter der geometrie und deren attribute....
dadurch: direkter einfluß aber auch direktes verständis der einzelnen elemente
\subsection{Konzept einer geeigneten Softwarelösung}
beispielhafte implementierung “datasynth”
anforderung an so eine software (beispiel1: journalist, beispiel2: privat/künstler)
-> datasynth als Erkundungstool für daten, als datenleser (ist das nicht auch eine kompetenz die man entwickeln sollte: daten lesen und verstehen zu können, also heisst es nicht auch heutzutage und in zukunft immer mehr: daten verstehen heißt die welt verstehen)

Warum visuelle Programmierung?

statt custom tools etc und immer wieder neu programmieren.....jetzt ein tool

komplexität hinter einem einfachen interface verstecken (klavier aus memo’s talk in london)

einordnung der software mit vergleichen zu anderen apps (grundlage sind funktionalität, also was kommt dabei raus) ist so ein ding zwischen statistiktool spss, 

\chapter{Methodik}
\section{Anforderungen}
1. konkretisierung der eigenen problemstellung
software erstellen, was muss sie können, wofür soll sie da sein
also warum z.b. visuelles programmieren usw.
2. entwicklung von hypothesen (oder definition von anforderungen)
was muss die software leisten können?
3. ausführliche beschreibung Untersuchungsmethodik und Vorgehensweise
????

auflistung:
\subsection{visuelle programmierung}
rapid prototyping / keine programmierkenntnisse erforderlich, ein durchschnittliches mathematisches verständnis reicht
\subsection{cross-platformness}
\subsection{freie Lizenz}
rails als beispiel für MIT Lizenz

\section{Analyse bestehender Lösungsansätze}
diskussion bisheriger lösungsansätze
vorstellen aller software die es bis jetzt gibt, siehe oben
evtl. (SWOT-)Analyse und damit aufzeigen der schwächen der anderen software
\subsection{vvvv}
\subsection{pd - Pure Data}
\subsection{prefuse}
\subsection{graphviz}
\subsection{...}
\section{Definition eines Anforderungskatalogs}
\chapter{Durchführung}
1. dokumentation der untersuchungsdurchführung (des Entwicklungsvorganges)
welche tools wurden wie angewendet, c++, of, git, boost, etc etc
2. darstellung der ergebnisse (verdeutlichung durch beispiele, erläuterung des erkenntnisfortschritts)
präsentation von viz mit der software?!
\section{Technologien}
\subsection{C++}
\subsection{openFrameworks}
\subsection{boost}
\section{wichtige Kernfunktionen}
auszüge aus quelltext mit erklärung
\section{Probleme bei der Entwicklung}
\chapter{Auswertung}
bewertung der ergebnisse vor dem hintergrund der hypothesen/problemstellung (Test des konstrukts und vergleich mit den Anforderungen
was kann die software liefern? wofür bildet sie nur die grundlage? was kann sie nicht?

2. ableiten von schlußfolgerungen
????
so eine software macht sinn, macht keinen sinn?
\section{Vorteile}
\section{Nachteile}

\chapter{Zusammenfassung}
1. zusammenfassung der ergebnisse, lesson learned
???
2. optional: persönliche bemerkungen, hinweis auf weiteren forschungsbedarf, ausblick in dei zukunft
was kannnoch alles an der Software getan werden?
\section{Ausblick}
\subsection{weitere Features}
erweiterung auf andere bereiche des visualisierungsprozesses


\section*{Viz und so}

Text über viz mit einer Fußnote\footnote{Streng genommen eine ziemlich sinnfreie.}

\section*{Querverweise}

Aber er hat einen coolen Verweis, siehe \cite[S.\,35--38]{Visualisierung}.
Mal sehen wie das in einer Fußnote\footnote{\cite{Visualisierung}} aussieht.

Um Querverweise nutzen zu können muss man z.B. auf Abschnitt~\ref{sec:testi} verweisen.

\bibliographystyle{dinat}
\bibliography{diplom}
\listoftables
\listoffigures
%\printnomenclature
\chapter{Anhang}
\end{document}

