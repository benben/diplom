\documentclass[a4paper, 12pt, headsepline]{scrreprt}

\usepackage[utf8]{inputenc}
\usepackage[T1]{fontenc}
\usepackage{setspace}
\onehalfspacing

\usepackage[ngerman]{babel}
\usepackage{natbib}
\usepackage{url} %correct url display in cites

\pagestyle{headings}

\typearea[current]{calc}

%add hyphenation later

\begin{document}
\title{Non-Standard-Datenvisualisierer mit openFrameworks}
\author{Benjamin Knofe}
\subject{Diplomarbeit}
\publishers{Hochschule für Technik, Wirtschaft und Kultur Leipzig}
\dedication{viele nette Leute}
\maketitle

\tableofcontents

\chapter{Einleitung}
\section{Ein Abschnitt}
\label{sec:testi}

In diesem Abschnitt steht viel toller Text.
Aber er hat einen coolen Verweis, siehe \cite[S.\,35--38]{Visualisierung}.
Mal sehen wie das in einer Fußnote\footnote{\cite{Visualisierung}} aussieht.

\section{Noch ein Abschnitt}
Mit viel anderem Text

\chapter{Viz}

\section{Viz und so}

Text über viz mit einer Fußnote\footnote{Streng genommen eine ziemlich sinnfreie.}

\section{Querverweise}

Um Querverweise nutzen zu können muss man z.B. auf Abschnitt~\ref{sec:testi} verweisen.

\bibliographystyle{dinat}
\bibliography{diplom}
\end{document}

